\documentclass{lipics-v2021}
%see https://submission.dagstuhl.de/documentation/authors

\title{A formalization of Dedekind domains and the class number}
\author{author name}{affiliation}{email}{orcid}{funding}

\usepackage{xcolor}
\usepackage{xspace}
\usepackage{listings}
\def\lstlanguagefiles{lstlean.tex}
\lstset{language=lean}

\newcommand{\lean}[1]{\texttt{#1}\xspace} % for writing Lean expressions
\newcommand{\OK}{\mathrm{O}_K}
\DeclareMathOperator{\Tr}{\mathrm{Tr}}
\newcommand{\N}{\mathbb{N}}
\newcommand{\Q}{\mathbb{Q}}
\newcommand{\Qbar}{\mathbb{\bar{Q}}}
\newcommand{\Z}{\mathbb{Z}}

\definecolor{keywordcolor}{rgb}{0.7, 0.1, 0.1}   % red
\definecolor{commentcolor}{rgb}{0.4, 0.4, 0.4}   % grey
\definecolor{symbolcolor}{rgb}{0.4, 0.4, 0.4}    % grey
\definecolor{sortcolor}{rgb}{0.1, 0.5, 0.1}      % green

\DeclareUnicodeCharacter{211A}{\ensuremath{\Q}}
\DeclareUnicodeCharacter{22A4}{\ensuremath{\top}}

\begin{document}

\maketitle

\begin{abstract}
We present our formalization of Dedekind domains and class numbers in the Lean prover. ...
\end{abstract}

\section{Introduction}

Our main achievement: if $K$ is a finite field extension of $\Q$, then the ring of integers $\OK$ is a Dedekind domain with finite class number.

Overview of the work:
\begin{itemize}
 \item Define dedekind domains
 \item show the definitions are equivalent to each other
 \item show a principal ideal domain is a Dedekind domain
 \item show the integral closure of a Dedekind domain in a finite (separable) field extension is a Dedekind domain
 \item define the class group
 \item show the class group is finite
\end{itemize}

\section{Number fields, global fields and rings of integers}

A number field is a finite field extension of $\Q$.
Number fields are a basic concept in algebraic number theory. Examples: ....
We formalized number fields as the following typeclass:
\begin{lstlisting}
class is_number_field (K : Type*) [field K] :=
[cz : char_zero K] [fd : finite_dimensional ℚ K]
\end{lstlisting}
The condition \lean{[cz : char\_zero K]} states that $K$ has characteristic zero, i.e. the canonical ring homomorphism $\N \to K$ is an embedding.
This implies that there is a $\Q$-algebra structure on $K$ (found by typeclass search), this gives the vector space structure used in the \lean{[fd : finite\_dimensional ℚ K]} hypothesis.

\subsection{Field extensions}

The definition of \lean{is\_number\_field} illustrates our treatment of field extensions.
In informal mathematics, a field $L$ containing a subfield $K$ is said to be a field extension $L / K$.
Often we encounter towers of field extensions: we might have that $\Q$ is contained in $K$, $K$ is contained in $L$, $L$ is contained in the algebraic closure $\bar{K}$ of $K$, and $\bar{K}$ is contained in $\C$.
We might formalize this situation by viewing $\Q$, $K$, $L$ and $\bar{K}$ to be sets of complex numbers $\C$ and defining field extensions as subset relations between these subfields.
This way, no coercions need to be inserted to map elements of one field into a larger field.
% I believe this is what mathcomp does; verify?
In type theory we cannot define $\Q$ as a subset of $\C$ since we need $\Q$ to define $\C$.
Thus, some coercion is always needed to go from the original definition of $\Q$ to its image in $\C$; and similar issues arise for other subfields that were not originally defined as such.
Moreover, we lose flexibility since we need to choose the top field at the start and cannot adjoin more elements when needed.

Instead, we parametrize our results over abritrary types $K$ and $L$ and represent the hypothesis ``$L$ is a field extension of $K$'' by taking an instance argument \lean{[algebra K L]}.
This provides us with a canonical ring homomorphism $\lean{algebra\_map K L} : K \to L$; this map is injective because $K$ and $L$ are fields.
In other words, field extensions are given by their canonical embeddings.

\subsection{Scalar towers}

The main drawback of allowing arbitrary embeddings is that we need to prove that these maps commute.
For example, we might start with a field extension $L / \Q$, then define a subfield $K$ of $L$,
giving a tower of extensions $L / K / \Q$.
In such a tower, the map $\Q \to L$ should be equal to the composition $Q \to K \to L$,
yet we cannot define this to be the case since the map $\Q \to L$ came earlier.

We formalize a result with hypothesis of the form ``$L / K / F$ is a tower of field extensions''
to be parametrized over all three maps \lean{[algebra F K]}, \lean{[algebra K L]} and \lean{[algebra F L]},
and take an additional parameter \lean{[is\_scalar\_tower F K L]} stating that these maps commute.

\lean{is\_scalar\_tower} works well.
Parameters are automatically inferred by the typeclass system in many cases,
such as when \lean{K : intermediate\_field F L}.

\subsection{Ring of integers}

A number ring is defined as a ring whose fraction field is a number field, the ring of integers $\OK$ is an important example.
The ring of integers in $K$ is defined as all $x : K$ that are the root of a monic polynomials with coefficients in $\Z$:
\begin{lstlisting}
def ring_of_integers (K : Type*) [field K] [is_number_field K] :=
integral_closure ℤ K
\end{lstlisting}

\section{Supporting definitions}

% I put this section in front because my notes later on refer back to definitions in this section. We could also give the informal overview first, then explain how we map this to formal maths.

\subsection{Subobjects}

We defined a subfield of $K$ as a subset of $K$ that contains $0$ and $1$ and is closed under addition, negation, multiplication, and taking inverses.

If $L$ is a field extension of $K$, we define an intermediate field as a subfield that is also a subalgebra: a subfield that contains the image of $\lean{algebra\_map K L}$.

If $K$ is an $R$-algebra and $x : K$, $x$ is integral if it is the root of a monic polynomial over $R$: there is a polynomial $p$ with leading coefficient $1$ and other coefficients in $R$, such that $p(x) = 0$.
The integral closure of $R$ in $K$ is the subalgebra containing all integral elements of $R$.
Elements of $K$ that are integral over the integral closure are also integral over $R$, explaining the word ``closure'' in the name.

\subsection{Fraction fields}

A fraction field $K$ of an integral domain $R$ is the smallest field that contains $R$ (or some other equivalent definitions).
The choice of $K$ is only unique up to isomorphism.
In particular, the generic construction of a fraction field of $\Z$ does not yield $\Q$.
One solution is to build a transfer tactic, the other is to state our theorems parametrized by $K$, along with a proof that $K$ is a fraction field of $R$.

The mathlib definition of fraction fields is based around the localization map. Let $R$ and $K$ be (commutative) rings with submonoid $P \subset R$, then $f : R \to K$ is a localization map if ..., expressed formally as the following structure:

The localization map $f$ endows $K$ with an $R$-algebra structure.

If the submonoid $P$ consists of all non-zero-divisors of $R$, we say that $f$ is a fraction map, and if $R$ is an integral domain, $K$ is a field. We call $K$ the fraction field of $R$.

The choice of $R$-algebra structure on $K$ is not unique, so we use a type synonym \lean{f.codomain}. This instructs the type class system to use the algebra instance derived from the localization map $f$.

In the following sections, let $f : R \to K$ be a fraction map.

\subsection{Fractional ideals}

When working with fraction fields, it is useful to extend the notion of $R$-ideals to fractional ideals: these are $R$-ideals divided by some $x : R$, or equivalently $R$-submodules $I$ of $K$ such that there is an $x : R$ with $x I \subseteq R$. The "$R$" in this statement is the image of $R$ in $K$, so our definition of fractional ideals depends on the fraction map $f$.

Submodules have a division but no inverses. This division lifts to dedekind domains, but dedekind domains have inverses for fractional ideals. This conflicts with the default definition of division, so we had to refactor the definition of \lean{group(\_with\_zero)} to include a field for division.

\subsection{Representing simple field extensions}

A number field $K$ is defined as a finite extension of $\Q$, or equivalently a field of the form $\Q(\alpha)$ for some algebraic number $\alpha$.
A field extension $L / K$ is called \emph{simple} if there is an $\alpha$ algebraic over $K$, called the \emph{primitive element}, such that $L = K(\alpha)$.
The primitive element theorem states that a finite, separable extension is simple; the converse holds if the primitive element $\alpha$ is separable.

The exact choice of these $\alpha$ and $K(\alpha)$ are underspecified in the mathematical literature.
We can find $K(\alpha)$ by adjoining the root of a polynomial: there is an irreducible polynomial $p \in K[X]$ such that $K[X] / p \simeq L$; we set $\alpha$ to be the image of $X$ in $K[X] / p$.
We can also take $\alpha : \bar{K}$, the algebraic closure of $K$, set $K(\alpha)$ to be the smallest subfield of $\bar{K}$ that contains $\alpha$ and the image of $K$, and have an equality $L = K(\alpha)$ as subfields of $\bar{K}$.
Similarly, we can take $\alpha : L$ and $K(\alpha)$ to be the smallest subfield of $L$ containing $\alpha$ and the image of $K$; then $L = K(\alpha)$ means that $K(\alpha)$, as a subfield of $L$, contains all elements of $L$.

Because $\alpha$ is algebraic the smallest subring containing $\alpha$ and $\Q$ will be a field, thus we can add two more representations, replacing ``smallest subfield'' with ``smallest subring''.
Moreover, all subfields/subrings containing $\Q$ are also $\Q$-algebras, so we can additionally replace ``subfield'' with ``intermediate field'' and ``subring'' with ``$\Q$-subalgebra''.

The ability to switch between these representations is important: sometimes $K$ and $L$ are fixed and we want an arbitrary $\alpha$; sometimes $\alpha$ is fixed and we want an arbitrary type representing $K(\alpha)$.
In Lean, these different constructions have already been formalized:
$K[X] / p$ is a type called \lean{adjoin\_root p},
$K(\alpha)$ as smallest $K$-subalgebra containing $\alpha$ is called \lean{subalgebra.adjoin $K$ \{$\alpha$\}}
and the smallest intermediate field containing $\alpha$ is \lean{intermediate\-\_field.adjoin $K$ \{$\alpha$\}}.
The primitive element theorem had been formalized as:
\begin{lstlisting}
theorem exists_primitive_element
  [finite_dimensional F E] [is_separable F E] :
  ∃ α : E, intermediate_field.adjoin F α = ⊤
\end{lstlisting}

Note that the choice of $\alpha$ (or the irreducible polynomial $p$) is not unique in general; both $3^\frac{1}{3}$ and $3^\frac{2}{3}$ generate $\Q(\sqrt[3]{3})$.
This means none of the above conditions provides a uniform way of reasoning about simple extensions:
if we use a predicate like ``finite, separable extension'' we cannot guarantee that the primitive element chosen for $K(\alpha)$ is indeed $\alpha$.
If we need to choose an $\alpha$ ahead of time and work in (any definition of) $K(\alpha)$, we need extra work to transfer the result across the isomorphism $K(\alpha) \simeq L$.

We chose instead to use a \emph{power basis} to represent simple field extensions, a basis of the form $1, x, x^2, \dots, x^{n-1} : A$ (viewing $A$ as an $R$-module).
We call $x$ the \emph{generator} and $n$ the \emph{dimension} of this power basis.
In Lean, we defined the following structure, bundling the information of a power basis:
\begin{lstlisting}
structure power_basis (R A : Type*) [comm_ring R] [ring A] [algebra R A] :=
(gen : S) (dim : ℕ)
(is_basis : is_basis R (λ (i : fin dim), gen ^ (i : ℕ)))
\end{lstlisting}

If $x : A$ is the generator of a power basis over $R$, it is also integral over $R$:
let $n$ be the dimension of the power basis, then $x^n : A$ can be written as $x^n = \sum_i c_i x^i$ for some coefficients $c_i : R$;
thus $p(X) = X^n - \sum_i c_i X^i$ is a polynomial with root $x$.
That $p$ has minimal degree, follows from the linear independence of the powers of $x$ up to $n$.
Conversely, for algebraic (and therefore integral) $\alpha$, $\Q(\alpha)$ has a power basis generated by $\alpha$.
This shows that the condition of having a power basis is equivalent to being a simple field extension.

With the \lean{power\_basis} structure, we have the ability to parametrize our results,
being able to choose the $K$ and $L$ in a simple field extension $L / K$,
or being able to choose the $\alpha$ generating $K(\alpha)$ (packaged up as \lean{power\_basis.gen\ pb}).
Specializing a result from an arbitrary $L$ with a power basis over $K$, to \lean{adjoin K \{$\alpha$\}} specifically, is a matter of applying the result to the power basis generated by $\alpha$, and rewriting $\lean{power\_basis.gen (adjoin.power\_basis K $\alpha$)} = \alpha$.


\section{Defining Dedekind Domains}

Initially we used 3 different structures to represent Dedekind domains: \lean{is\_dedekind\_domain}, \lean{is\_dedekind\_domain\_iff} and \lean{is\_dedekind\_domain\_dvr}, defined as follows:
...

These different structures allowed us to do our work in parallel. This meshed well with the approach in mathlib of favouring short, complete, individual contributions over including projects. (This is also how the Linux kernel gets developed!)

In parallel we could then work on providing instances of the \lean{is\_dedekind\_domain} typeclass, proving the equivalences between the structures, and using the Dedekind domain definition as a hypothesis.

Discuss the \lean{not\_is\_field} assumption.

\section{Equivalence of the definitions}

In this section, we describe how we proved that the 3 definitions of Dedekind domain are equivalent.

\section{Principal ideal domains are Dedekind}

This is not a long proof, so it can be a good demonstration of our definitions.

\section{Ring of integers are Dedekind domains}

We prove a stronger statement: if $R$ is a Dedekind domain with fraction field $K$, and $L$ is a finite separable extension of $K$, then the integral closure of $R$ in $L$ is a Dedekind domain.

The integral closure is immediately integrally closed.

The integral closure has dimension at most one: some going-up theory needed. Let $I$ be a prime ideal in the closure, then $I \cap R$ is a prime ideal not equal to $R$ itself, which is maximal. If $I < J$ then we find an integral $x \in J \setminus I$, and the smallest nonzero coefficient of $x$'s minimal polynomial is in $J \cap R$.

The integral closure is noetherian: Because the trace form is nondegenerate and $L / K$ is finite-dimensional, there is a basis $b$ for $L / K$ which is mapped to a dual basis $b'$, such that the $i$'th coordinate of $x$ according to $b'$ is exactly the trace of $b_i x$. Let $x$ be integral over $R$, because the trace of $x$ is equal (in an algebraic closure of $L$) to the sum of roots of the characteristic polynomial of $x$, each of these roots is integral so the sum is integral too; therefore the trace of $x$ is integral over $R$. Now we suppose that all $b_i$ are integral (we can multiply the whole of $b$ by a constant until all $b_i$ are integral), and let $x : L$ be integral over $R$: then $\sum_i \Tr(b_i x) b'_i = \sum_i x_i b'_i = x$, and each $\Tr(b_i x) : K$ is integral over the Dedekind domain $R$, so it is an element of $R$. We conclude that the integral closure of $R$ in $L$ is contained in the $R$-span of $b'_i$, so it is noetherian.

When working with the trace of $x$, main difficulties that we resolved were: many different ways of arriving at K(x) (fixed using power basis); switching between many different extensions (fixed using \lean{is\_scalar\_tower}).

If $R$ is a principal ideal domain, such as $\Z$, we can strengthen the conclusion more: there is a linear independent set of integral elements spanning the integral closure. This is one part of the structure theorem of finitely generated modules.

\section{Class number}

The class group is the quotient \lean{units (fractional\_ideal f)} modulo the principal fractional ideals, or equivalently the ideals of $R$ (except $0$) modulo the elements of $R$ (except $0$).

We are interested in the class group because ...

An important property of the ring of integers in a number field is that the class group is finite. The number of elements of the class group is called the class number.

We prove that the class group is finite using a simplified form of the traditional proof, ...

\end{document}
