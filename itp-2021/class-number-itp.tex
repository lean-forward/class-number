\documentclass{beamer}
\usetheme[nofirafonts]{focus}
\usepackage{tikz}
\usepackage{xspace}

\usepackage{fontspec}
\setmainfont[ItalicFont={* Italic}, BoldFont={* Bold}]{TeX Gyre Heros}
\setmonofont{DejaVu Sans Mono}[SizeFeatures={Size=9}]

\usepackage{listings}
\def\lstlanguagefiles{lstlean.tex}
\lstset{language=lean,backgroundcolor=\color[rgb]{0.9,0.9,0.9}}
\definecolor{keywordcolor}{rgb}{0.7, 0.1, 0.1}   % red
\definecolor{commentcolor}{rgb}{0.4, 0.4, 0.4}   % grey
\definecolor{symbolcolor}{rgb}{0.4, 0.4, 0.4}    % grey
\definecolor{sortcolor}{rgb}{0.1, 0.5, 0.1}      % green

\usepackage{parskip}
\setlength{\parskip}{\baselineskip}

\newcommand{\C}{\mathbb{C}}
\newcommand{\lean}[1]{\texttt{#1}\xspace}
\newcommand*{\OK}[1][K]{\mathcal{O}_{#1}}
\newcommand*{\Cl}{\mathcal{C}\kern-.075em l}
\newcommand*{\Fq}[1][q]{\mathbb{F}_{#1}}
\DeclareMathOperator{\Tr}{Tr}
\newcommand{\mathlib}{\textsf{mathlib}\xspace}
\newcommand{\N}{\mathbb{N}}
\newcommand{\R}{\mathbb{R}}
\newcommand{\pow}{\textasciicircum\xspace}
\newcommand{\Q}{\mathbb{Q}}
\newcommand{\Z}{\mathbb{Z}}
\DeclareMathOperator{\Frac}{Frac}

% \DeclareUnicodeCharacter{03B1}{\ensuremath{\alpha}}
% \DeclareUnicodeCharacter{03C3}{\ensuremath{\sigma}}
% \DeclareUnicodeCharacter{2081}{\ensuremath{_1}}
% \DeclareUnicodeCharacter{2090}{\ensuremath{_a}}
% \DeclareUnicodeCharacter{2097}{\ensuremath{_l}}
% \DeclareUnicodeCharacter{211A}{\ensuremath{\Q}}
% \DeclareUnicodeCharacter{2124}{\ensuremath{\Z}}
% \DeclareUnicodeCharacter{2211}{\ensuremath{\sum}}
% \DeclareUnicodeCharacter{2264}{\ensuremath{\l}}
% \DeclareUnicodeCharacter{2286}{\ensuremath{\subseteq}}
% \DeclareUnicodeCharacter{22A4}{\ensuremath{\top}}
% \DeclareUnicodeCharacter{22A5}{\ensuremath{\bot}}

\title{A formalization of Dedekind domains and class groups of global fields}
\author{Anne Baanen \\ Sander R. Dahmen \\ Ashvni Narayanan \\ Filippo A. E. Nuccio}
\titlegraphic{\includegraphics[scale=.1]{lf_vu.png}\includegraphics[scale=.1]{Logo_UJM.jpg}}
%\titlegraphic{\includegraphics[scale=.2]{vu.png}}
%\institute{Institute Name \\ Institute Address}
\date{ITP, June 2021}

\begin{document}
\begin{frame}
	\maketitle
\end{frame}

\begin{frame}{Introduction}
	Our project is the first formalization of several essential notions of
	\alert{algebraic number theory}, in the Lean 3 prover as part of \mathlib.
	% Example application: 26 is the only integer one bigger than a square and one smaller than a cube.

	Goal: lay a useful foundation for theory-building.

	Developing with \mathlib means updating your code regularly\\
	in exchange for frequent new results and improvements.
\end{frame}

\begin{frame}{Background: global fields}
	A \alert{number field} is finite dimensional as a $\Q$-vector space,\\
	of the form $\Q(\alpha)$ for some algebraic $\alpha$.

\pause
	Each number field $K$ contains a \alert{ring of integers} $\OK$\\
	mirroring the way $\Q$ contains $\Z$.\\
	\textbf{Example}: the Gaussian integers $\Z[i]$ inside $\Q(i)$.

\pause
	A \alert{global field} is either a \alert{number field} or a\\
	\alert{function field}: finite extension of a field of rational functions $\Fq(t)$.

	Function fields also have a ring of integers, mirroring $\Fq[q][t] \subset \Fq(t)$.

\pause
	\textbf{Theorem}: rings of integers are \alert{Dedekind domains}.
\end{frame}

\begin{frame}{Background: class number}
	\alert{Fractional ideals} extend (integral) ideals with division by scalars:\\
	a fractional ideal is of the form $\frac{1}{b} I$ (with $b \ne 0$).\\
	\alert{Fractional ideals are not automatically invertible!}

\pause
	\textbf{Theorem}: Dedekind domain $\iff$ fractional ideals $\ne 0$ are invertible.

\pause
	Principal fractional ideals $\left\langle \frac{a}{b} \right\rangle = \frac{a}{b} \OK$ for $\frac{a}{b} \in K$ form a subgroup\\
	of the fractional ideals; the quotient is the \alert{class group} $\Cl_{\OK}$.

	\textbf{Theorem}: if $\OK$ is a ring of integers, $\Cl_{\OK}$ is a finite abelian group.\\
	The \alert{class number} of $K$ is the cardinality of $\Cl_{\OK}$.

\pause
	\textbf{Theorem}: A Dedekind domain is a UFD $\iff$ it is a PID\\
	$\iff$ $\Cl_{\OK}$ is trivial $\iff$ class number of $K$ $= 1$.
\end{frame}

\begin{frame}[fragile]{Number fields; field extensions}
	\mathlib typically uses typeclasses for algebraic structures, e.g.

\begin{lstlisting}
class is_number_field (K : Type*) [field K] : Prop :=
[cz : char_zero K] [fd : finite_dimensional ℚ K]
\end{lstlisting}

	Typeclass inference automates the implications \lean{char\_zero K $\to$ algebra ℚ K $\to$ module ℚ K}
	required for \lean{finite\_dimensional ℚ K}.

\pause

	A \alert{field extension} $L / K$ is represented in \mathlib by an instance \lean{[algebra K L]}
	giving the canonical inclusion map \lean{algebra\_map K L}.

\pause
	A tower $L / K / F$ is given by inclusions \lean{[algebra F K] [algebra K L] [algebra F L]}
	and an instance \lean{[is\_scalar\_tower F K L]} stating the maps commute.\\
	Coherence proof obligations are automated through typeclass search.
\end{frame}

%\begin{frame}{Field of fractions}
%	Several constructions of the \alert{field of fractions} $\Frac(R)$:\\
%	quotient type of $R \times (R \setminus \{0\})$, subtype of $R \times R$, $R$ itself (if $R$ is a field), an optimized implementation of $\Q = \Frac(\Z)$\dots
%
%	These are isomorphic but not equal: how to reason uniformly?
%
%	\pause
%	Use the unique property: $f : R \to K$ is a \alert{fraction map} if all injective maps $g : R \to A$ with $g(x)$ a unit, can be lifted to $K \to A$.
%\end{frame}

\begin{frame}{Representing $\Q(\alpha)$}
	Number fiels have the form $\Q(\alpha)$, where $\alpha$ is algebraic:\\
	the minimal polynomial $f_\alpha \in \Q[x]$ is irreducible and $f_\alpha(\alpha) = 0$.

	Several constructions of $\Q(\alpha)$: subtype of $\C$, quotient type $\Q[x] / P$, ...\\
	These are isomorphic but not equal: how do we reason uniformly?

\pause
	We used the \alert{power basis}: $\Q(\alpha)$ has a $\Q$-basis $1, \alpha, \dots, \alpha^{n - 1}$.
	\textbf{Theorem}: each construction of $\Q(\alpha)$ is a field with power basis generated by $\alpha$.
\end{frame}

\begin{frame}[fragile]{Dedekind domains}
	We defined Dedekind domains as integral domains $R$ with an \lean{is\_dedekind\_domain R} instance:
\begin{lstlisting}
class is_dedekind_domain (R : Type*) [integral_domain R] : Prop :=
(to_is_noetherian_ring : is_noetherian_ring R)
(dimension_le_one : ∀ (P : ideal R), P ≠ ⊥ →
  is_prime P → is_maximal P)
(is_integrally_closed :
  integral_closure R (fraction_ring R) = ⊥)
\end{lstlisting}

\end{frame}

\begin{frame}{Fractional ideals}
	We formalized fractional ideals of $R$ as\\
	$R$-submodules $I$ of $\Frac(R)$ such that $\exists a : R, a I \subseteq R$.

	Fractional ideals have a semiring structure (like submodules):\\
	\begin{itemize}
		\item $0 = \{0\}$
		\item $1 = \{x \mid x \in R\}$
		\item $I + J = \{x + y \mid x \in I, y \in J\}$
		\item $I * J$ is generated by $x * y$, $x \in I$, $y \in J$
		\item $x \in I / J \iff \forall y \in J, x * y \in I$
	\end{itemize}

\pause
	\textbf{Theorem}: $I * (1 / I) = 1$ for all $I \ne 0$ iff $R$ is a Dedekind domain.
\end{frame}

\begin{frame}{Re-defining division}
	The \lean{group\_with\_zero} typeclass used to define $x / y := x * y^{-1}$.\\
	For fractional ideals we want to define $I^{-1} := 1 / I$.
	How to deal with this circularity?

	\pause
	Solution: turn defeq into propositional equality by
	adding a new operation $(/)$ to \lean{group}(\lean{\_with\_zero})
	and an axiom $x / y = x * y^{-1}$.

	This required about 500 changes in \mathlib.
\end{frame}

\begin{frame}{Dedekind domain theorems}
	\textbf{Theorem}: $I * (1 / I) = 1$ for all $I \ne 0$ iff $R$ is a Dedekind domain.

	Difficulties:
	\begin{itemize}
		\item Showing $x \in I * J$ implies $x = \sum_k y_k z_k$ for $y_k \in I$, $z_k \in J$.
		\item Coercions: $I$ can be an integral ideal or set $\subseteq R$ or a fractional ideal or submodule or set $\subseteq \Frac(R)$.
	\end{itemize}

\pause
	\textbf{Theorem}: Principal ideal domains are Dedekind domains.\\
	\textbf{Corollary}: $\Z$ and $\Fq[q][t]$ are Dedekind domains.
\end{frame}

\begin{frame}{Rings of integers are Dedekind domains}
	\textbf{Theorem}: The integral closure of a Dedekind domain $R$ in a finite separable extension $K / \Frac(R)$ is a Dedekind domain.\\
	\textbf{Corollary}: Rings of integers, closures of PIDs in finite separable extensions, are Dedekind domains.

\pause
	``Accidental'' collaboration with the Berkeley Galois theory group:
	\begin{itemize}
		\setlength\itemsep{-0.5\baselineskip}
		\item We defined \lean{intermediate\_field}
		\item They used it to formalize the primitive element theorem
		\item We used that to show finite separable field extensions have a power basis
		\item They used that to show conjugate roots of $\alpha$ correspond to images $\sigma(\alpha)$ for $\sigma : F(\alpha) \to K$ fixing $F$
		\item We used that to show the \alert{trace form} is nondegenerate
	\end{itemize}
\end{frame}

\begin{frame}[fragile]{Finiteness of the class group}
	\textbf{Theorem}: If $K$ is a global field, the class group of $\OK$ is finite.

	Typical proofs use Minkowski's lattice point theorem for number fields,\\
	extending this to function fields is complicated.

\pause
	We introduced a new notion of \alert{admissible absolute value},\\
	and proved if $\lean{abs} : R \to \Z$ is admissible,\\
	this intermediate step in the classical proof holds:
\begin{lstlisting}
theorem exists_mem_finset_approx'
  (a b : integral_closure R L) :=
  ∃ (q : integral_closure R L) (r ∈ finset_approx L f abs),
  abs_norm f abs (r • a - q * b) < abs_norm f abs b
\end{lstlisting}
\end{frame}

\begin{frame}[fragile]{Finiteness of the class group}
	After formalizing the remainder of the classical proof,
	it remained to find admissible absolute values.

	For $\Z$, the usual absolute value is admissible.\\
	For $\Fq[q][t]$, $\lvert f\rvert_{\deg}:=q^{\deg f}$ is admissible.

\vspace{\baselineskip}
\pause
\begin{lstlisting}
def class_group (f : fraction_map R K) :=
quotient_group.quotient (to_principal_ideal f).range

noncomputable def number_field.class_number (K : Type*)
  [field K] [is_number_field K] : ℕ :=
card (class_group (ring_of_integers.fraction_map K))
\end{lstlisting}
\end{frame}

\begin{frame}{Conclusions}
	Total contribution: $\pm$ 5000 lines of project-specific code, $\pm$ 2500 lines background work.\\
	(Difficult to quantify exactly due to tight integration with \mathlib.)

\pause
	Rules of thumb for our work:
	\begin{itemize}
		\item Parametrize (\lean{is\_scalar\_tower}, \lean{power\_basis}, ...) instead of choosing a canonical construction.
		\item Refactoring allows deep integration between different viewpoints.
		\item Contribute quickly and often, so others do your work for you.
	\end{itemize}

\end{frame}
\end{document}
